\documentclass[]{article}
\usepackage{lmodern}
\usepackage{amssymb,amsmath}
\usepackage{ifxetex,ifluatex}
\usepackage{fixltx2e} % provides \textsubscript
\ifnum 0\ifxetex 1\fi\ifluatex 1\fi=0 % if pdftex
  \usepackage[T1]{fontenc}
  \usepackage[utf8]{inputenc}
\else % if luatex or xelatex
  \ifxetex
    \usepackage{mathspec}
  \else
    \usepackage{fontspec}
  \fi
  \defaultfontfeatures{Ligatures=TeX,Scale=MatchLowercase}
\fi
% use upquote if available, for straight quotes in verbatim environments
\IfFileExists{upquote.sty}{\usepackage{upquote}}{}
% use microtype if available
\IfFileExists{microtype.sty}{%
\usepackage{microtype}
\UseMicrotypeSet[protrusion]{basicmath} % disable protrusion for tt fonts
}{}
\usepackage[margin=1in]{geometry}
\usepackage{hyperref}
\hypersetup{unicode=true,
            pdftitle={Besemah stress},
            pdfauthor={Bradley McDonnell},
            pdfborder={0 0 0},
            breaklinks=true}
\urlstyle{same}  % don't use monospace font for urls
\usepackage{color}
\usepackage{fancyvrb}
\newcommand{\VerbBar}{|}
\newcommand{\VERB}{\Verb[commandchars=\\\{\}]}
\DefineVerbatimEnvironment{Highlighting}{Verbatim}{commandchars=\\\{\}}
% Add ',fontsize=\small' for more characters per line
\usepackage{framed}
\definecolor{shadecolor}{RGB}{248,248,248}
\newenvironment{Shaded}{\begin{snugshade}}{\end{snugshade}}
\newcommand{\KeywordTok}[1]{\textcolor[rgb]{0.13,0.29,0.53}{\textbf{#1}}}
\newcommand{\DataTypeTok}[1]{\textcolor[rgb]{0.13,0.29,0.53}{#1}}
\newcommand{\DecValTok}[1]{\textcolor[rgb]{0.00,0.00,0.81}{#1}}
\newcommand{\BaseNTok}[1]{\textcolor[rgb]{0.00,0.00,0.81}{#1}}
\newcommand{\FloatTok}[1]{\textcolor[rgb]{0.00,0.00,0.81}{#1}}
\newcommand{\ConstantTok}[1]{\textcolor[rgb]{0.00,0.00,0.00}{#1}}
\newcommand{\CharTok}[1]{\textcolor[rgb]{0.31,0.60,0.02}{#1}}
\newcommand{\SpecialCharTok}[1]{\textcolor[rgb]{0.00,0.00,0.00}{#1}}
\newcommand{\StringTok}[1]{\textcolor[rgb]{0.31,0.60,0.02}{#1}}
\newcommand{\VerbatimStringTok}[1]{\textcolor[rgb]{0.31,0.60,0.02}{#1}}
\newcommand{\SpecialStringTok}[1]{\textcolor[rgb]{0.31,0.60,0.02}{#1}}
\newcommand{\ImportTok}[1]{#1}
\newcommand{\CommentTok}[1]{\textcolor[rgb]{0.56,0.35,0.01}{\textit{#1}}}
\newcommand{\DocumentationTok}[1]{\textcolor[rgb]{0.56,0.35,0.01}{\textbf{\textit{#1}}}}
\newcommand{\AnnotationTok}[1]{\textcolor[rgb]{0.56,0.35,0.01}{\textbf{\textit{#1}}}}
\newcommand{\CommentVarTok}[1]{\textcolor[rgb]{0.56,0.35,0.01}{\textbf{\textit{#1}}}}
\newcommand{\OtherTok}[1]{\textcolor[rgb]{0.56,0.35,0.01}{#1}}
\newcommand{\FunctionTok}[1]{\textcolor[rgb]{0.00,0.00,0.00}{#1}}
\newcommand{\VariableTok}[1]{\textcolor[rgb]{0.00,0.00,0.00}{#1}}
\newcommand{\ControlFlowTok}[1]{\textcolor[rgb]{0.13,0.29,0.53}{\textbf{#1}}}
\newcommand{\OperatorTok}[1]{\textcolor[rgb]{0.81,0.36,0.00}{\textbf{#1}}}
\newcommand{\BuiltInTok}[1]{#1}
\newcommand{\ExtensionTok}[1]{#1}
\newcommand{\PreprocessorTok}[1]{\textcolor[rgb]{0.56,0.35,0.01}{\textit{#1}}}
\newcommand{\AttributeTok}[1]{\textcolor[rgb]{0.77,0.63,0.00}{#1}}
\newcommand{\RegionMarkerTok}[1]{#1}
\newcommand{\InformationTok}[1]{\textcolor[rgb]{0.56,0.35,0.01}{\textbf{\textit{#1}}}}
\newcommand{\WarningTok}[1]{\textcolor[rgb]{0.56,0.35,0.01}{\textbf{\textit{#1}}}}
\newcommand{\AlertTok}[1]{\textcolor[rgb]{0.94,0.16,0.16}{#1}}
\newcommand{\ErrorTok}[1]{\textcolor[rgb]{0.64,0.00,0.00}{\textbf{#1}}}
\newcommand{\NormalTok}[1]{#1}
\usepackage{longtable,booktabs}
\usepackage{graphicx,grffile}
\makeatletter
\def\maxwidth{\ifdim\Gin@nat@width>\linewidth\linewidth\else\Gin@nat@width\fi}
\def\maxheight{\ifdim\Gin@nat@height>\textheight\textheight\else\Gin@nat@height\fi}
\makeatother
% Scale images if necessary, so that they will not overflow the page
% margins by default, and it is still possible to overwrite the defaults
% using explicit options in \includegraphics[width, height, ...]{}
\setkeys{Gin}{width=\maxwidth,height=\maxheight,keepaspectratio}
\IfFileExists{parskip.sty}{%
\usepackage{parskip}
}{% else
\setlength{\parindent}{0pt}
\setlength{\parskip}{6pt plus 2pt minus 1pt}
}
\setlength{\emergencystretch}{3em}  % prevent overfull lines
\providecommand{\tightlist}{%
  \setlength{\itemsep}{0pt}\setlength{\parskip}{0pt}}
\setcounter{secnumdepth}{0}
% Redefines (sub)paragraphs to behave more like sections
\ifx\paragraph\undefined\else
\let\oldparagraph\paragraph
\renewcommand{\paragraph}[1]{\oldparagraph{#1}\mbox{}}
\fi
\ifx\subparagraph\undefined\else
\let\oldsubparagraph\subparagraph
\renewcommand{\subparagraph}[1]{\oldsubparagraph{#1}\mbox{}}
\fi

%%% Use protect on footnotes to avoid problems with footnotes in titles
\let\rmarkdownfootnote\footnote%
\def\footnote{\protect\rmarkdownfootnote}

%%% Change title format to be more compact
\usepackage{titling}

% Create subtitle command for use in maketitle
\newcommand{\subtitle}[1]{
  \posttitle{
    \begin{center}\large#1\end{center}
    }
}

\setlength{\droptitle}{-2em}

  \title{Besemah stress}
    \pretitle{\vspace{\droptitle}\centering\huge}
  \posttitle{\par}
  \subtitle{A basic example}
  \author{Bradley McDonnell}
    \preauthor{\centering\large\emph}
  \postauthor{\par}
      \predate{\centering\large\emph}
  \postdate{\par}
    \date{12/29/2018}

\usepackage{booktabs}
\usepackage{longtable}
\usepackage{array}
\usepackage{multirow}
\usepackage[table]{xcolor}
\usepackage{wrapfig}
\usepackage{float}
\usepackage{colortbl}
\usepackage{pdflscape}
\usepackage{tabu}
\usepackage{threeparttable}
\usepackage{threeparttablex}
\usepackage[normalem]{ulem}
\usepackage{makecell}

\begin{document}
\maketitle

\begin{verbatim}
## # A tibble: 6 x 13
##   WordOrder Word  Participant Position Vowel Weight Focus   Final Duration
##       <int> <fct> <fct>       <fct>    <fct> <fct>  <fct>   <fct>    <dbl>
## 1         1 tutus S1          penult   u     light  in-foc~ final   0.0754
## 2         1 tutus S1          ultima   u     light  in-foc~ final   0.189 
## 3         3 tatap S1          penult   a     light  in-foc~ final   0.0697
## 4         3 tatap S1          ultima   a     light  in-foc~ final   0.149 
## 5         4 pipis S1          penult   i     light  in-foc~ final   0.0799
## 6         4 pipis S1          ultima   i     light  in-foc~ final   0.213 
## # ... with 4 more variables: F0 <dbl>, H1H2 <dbl>, Intensity <dbl>,
## #   SpectralBalance <dbl>
\end{verbatim}

\section{Introduction}\label{introduction}

There has been much disagreement over the status of word-level stress in
the languages of western Indonesia, particularly in regards to
well-known varieties of Malay, such as Standard Indonesian (van Zanten,
Goedemans, and Pacilly 2003).\footnote{Text from this example is lifted
  directly from McDonnell \& Turnbull (2018).} Word-level stress has
been claimed to fall predictably on the penultimate syllable unless it
contains a schwa in which case it falls on the ultima (Adisasmito-Smith
and Cohn 1996). However, since the late 1990s, an increasing number of
studies have questioned this position for Standard Indonesian (van
Zanten and van Heuven 2004), and more recent studies have pointed out
the complications associated with studies of Standard Indonesian due to
significant influence from substrate languages like Javanese (Goedemans
and van Zanten 2007).

The present study is designed to tease apart different factors relating
to prosodic prominence in Besemah. It does so by investigating the
realization of common acoustic correlates of stress (f\(_0\), duration,
intensity, and spectral tilt) within target words that vary in their
sentence position (sentence-medial vs.~sentence-final position) and
information status (`in focus' vs. `out of focus').

\subsection{Besemah word level stress}\label{besemah-word-level-stress}

Like many of the languages of western Indonesia, Besemah has received
little attention since the Dutch colonial period (McDonnell 2008). The
only study of Besemah prosody tentatively concluded that word-level
stress in Besemah falls on the final syllable of the word and is cued by
increased intensity (McDonnell 2016). However, there are several
complications to this analysis, and word-level stress appears to be
affected by the presence of final L boundary tones.

\section{Methods}\label{methods}

This section describe the methodology used in this study.

\subsection{Design}\label{design}

Target words were collected using an information gap task. This task
involved two talkers: a confederate who asked questions and a naive
participant who answered them. They sat facing each other, each able to
see only their own laptop screen. Both the confederate and participant
could see a question on the top of the screen that the confederate was
to ask, but only the participant's screen displayed the answer.The task
was for the confederate to ask the question and the participant to
provide an answer in a complete sentence modeled upon the question. The
confederate then circled the answer on the paper. One female native
speaker of Besemah acted as the confederate for all participants.

\subsection{Materials}\label{materials}

Question-answer pairs were constructed which varied in information
status (target word `in focus' or `out of focus') and sentence position
(target word sentence-medial or sentence-final). The type and structure
of the question depended upon the combination of its sentence position
and information status. Question type differed based upon whether the
target word was `in focus' or `out of focus.'\footnote{`Focus', as a
  term of art in linguistics, is fraught with definitional difficulties.
  In our `in focus' condition, the target word is the answer to the
  current question under discussion (Roberts 2012). In the `out of
  focus' condition, on the other hand, the target word is given, not new
  (Schwarzschild 1999). This definition may not match exactly with the
  definitions used by other researchers, and we invite the reader to
  mentally substitute their favorite terms that fit their proclivities
  if they are so inclined.} The sentence position simply altered the
shape of the question so that the target word would occur in
sentence-medial or sentence-final position. The table below presents
example question-answer pairs for the `in focus' condition. In this
table the target word \emph{susu} `milk' is used as an example and
displayed in boldface.

\begin{longtable}[]{@{}lll@{}}
\caption{Information status condition}\tabularnewline
\toprule
\textbf{Position} & & \textbf{Question/Answer}\tabularnewline
\midrule
\endfirsthead
\toprule
\textbf{Position} & & \textbf{Question/Answer}\tabularnewline
\midrule
\endhead
final & Q: & \emph{Sebelah kanan kate ape?}\tabularnewline
& & `On the right side is what word?'\tabularnewline
& A: & \emph{Sebelah kanan kate \textbf{susu}.}\tabularnewline
& & `On the right side is \textbf{milk}.'\tabularnewline
medial & Q: & \emph{Kate ape sebelah kanan?}\tabularnewline
& & `What word is on the right side?'\tabularnewline
& A: & \emph{Kate} \textbf{\emph{susu}} \emph{sebelah
kanan.}\tabularnewline
& & `The word \textbf{milk} is on the right side.'\tabularnewline
\bottomrule
\end{longtable}

For stimuli that seek to collect target words that are `out of focus',
the confederate asked a different set of questions. Examples are shown
in the table below. Unlike the `in focus' condition, the same target
word was mentioned four times in immediately preceding turns. This
mentioning of the target word served to keep the word as discourse-given
and thus `out of focus'. Further, each time the word appeared for the
first time in a block it was introduced by asking the question
\emph{Titu kate ape?} `What word is this?'. The target word in this
question is `in focus' in this utterance and not considered in the
analysis. However, this allowed the confederate to ask questions about
the target word without it being `in focus'. Thus, the subsequent
repetitions of the target word ask where the target word appears on the
screen, the top, bottom, left or right side. In these four questions,
the target word is `out of focus' while the word describing its position
is `in focus'. This table presents example question-answer pairs for the
`out of focus' condition.

The twelve target words are shown in the table below. Syllables in
Besemah are maximally CVC with some restrictions on the coda consonants
and words are most commonly bisyllabic. The words feature all four
phonotactically legal combinations of light (CV) and heavy (CVC)
syllables. Each of the four vowels /i, u, a, ə/ was present and matched
in both syllables to allow for easier intra-word comparisons.
Phonotactically, the high vowels /i, u/ show no restrictions, but /a/
does not occur word-finally and /ə/ does not occur in final closed
syllables.

\begin{longtable}[]{@{}rrl@{}}
\caption{Target words used in constructing question-answer
pairs.}\tabularnewline
\toprule
\textbf{Phoneme} & \textbf{Word} &\tabularnewline
\midrule
\endfirsthead
\toprule
\textbf{Phoneme} & \textbf{Word} &\tabularnewline
\midrule
\endhead
/i/ & /titi/ & `to cross over'\tabularnewline
& /pipis/ & `to pulverize'\tabularnewline
& /tʃiŋki/ & `must have'\tabularnewline
& /sintiŋ/ & `crooked'\tabularnewline
/u/ & /susu/ & `milk'\tabularnewline
& /tutus/ & `to pound'\tabularnewline
& /tuŋku/ & `hearth'\tabularnewline
& /tuntun/ & `to watch'\tabularnewline
/a/ & --- &\tabularnewline
& /tatap/ & `to touch'\tabularnewline
& --- &\tabularnewline
& /pantas/ & `to be fitting'\tabularnewline
/ə/ & /tʃətə/ & `to be exact'\tabularnewline
& --- &\tabularnewline
& /təmpə/ & `to forge'\tabularnewline
& --- &\tabularnewline
\bottomrule
\end{longtable}

\subsection{Procedure}\label{procedure}

There were four experimental blocks that combined the possible
combinations of information status and sentence position:

\begin{itemize}
\tightlist
\item
  Block A: sentence-medial, `in focus'
\item
  Block B: sentence-final, `in focus'
\item
  Block C: sentence-medial, `out of focus'
\item
  Block D: sentence-final, `in focus'
\end{itemize}

All recordings were made on a Marantz PMD-670 solid state recorder with
a sampling frequency of 48kHz and stored as wav files. The participant
was recorded on the left channel with a Shure WH-30 headset microphone
while the confederate was recorded on the right channel with a Shure
SM10A headset microphone.

Each target word was repeated four times for each of the four blocks (12
words \(\times\) 4 repetitions \(\times\) 4 blocks = 192 tokens). The
presentation of these words was randomized and the presentation of each
block was also randomized.

\subsection{Participants}\label{participants}

The experiment was conducted by the first author over several days in
March\textasciitilde{}2015 in the Besemah village of Karang Tanding. The
experiment was conducted entirely in Besemah with no interlanguage. The
six female participants in this study are native speakers of Besemah and
reside in or near the village of Karang Tanding in South Sumatra
province. They range in age from 19 to 30 years old.

\section{Acoustic correlates of
stress}\label{acoustic-correlates-of-stress}

This section takes a preliminary look at the acoustic correlates of
stress in Besemah.

\subsection{Duration}\label{duration}

Duration in the final syllable appears to be longer compared to duration
of the penultimate syllable. It is noteworthy, however, that while
duration may be the most robust correlate for stress, it is confounded
with word-final lengthening.

\begin{Shaded}
\begin{Highlighting}[]
\CommentTok{# A simple boxplot of intensity}
\NormalTok{dur <-}\StringTok{ }\KeywordTok{ggplot}\NormalTok{(pse_stress, }\KeywordTok{aes}\NormalTok{(Position, Duration))}
\NormalTok{dur }\OperatorTok{+}\StringTok{ }\KeywordTok{geom_boxplot}\NormalTok{() }\OperatorTok{+}\StringTok{ }\KeywordTok{coord_flip}\NormalTok{() }\OperatorTok{+}\StringTok{ }\KeywordTok{scale_x_discrete}\NormalTok{(}\DataTypeTok{limits =} \KeywordTok{rev}\NormalTok{(}\KeywordTok{levels}\NormalTok{(pse_stress}\OperatorTok{$}\NormalTok{Position)))}
\end{Highlighting}
\end{Shaded}

\includegraphics{phonetics_paper_files/figure-latex/duration-1.pdf}

As expected, when the word is in the final position in the sentence, the
duration is longer in the ultimae. The penult appears to be fairly
similar whether or not the word is sentence final or not.

\begin{Shaded}
\begin{Highlighting}[]
\CommentTok{# A simple boxplot of intensity}
\NormalTok{dur <-}\StringTok{ }\KeywordTok{ggplot}\NormalTok{(pse_stress, }\KeywordTok{aes}\NormalTok{(Position, Duration))}
\NormalTok{dur }\OperatorTok{+}\StringTok{ }\KeywordTok{geom_boxplot}\NormalTok{() }\OperatorTok{+}\StringTok{ }\KeywordTok{facet_wrap}\NormalTok{(}\OperatorTok{~}\NormalTok{Final, }\DataTypeTok{ncol =} \DecValTok{1}\NormalTok{) }\OperatorTok{+}\StringTok{ }\KeywordTok{coord_flip}\NormalTok{() }\OperatorTok{+}\StringTok{ }\KeywordTok{scale_x_discrete}\NormalTok{(}\DataTypeTok{limits =} \KeywordTok{rev}\NormalTok{(}\KeywordTok{levels}\NormalTok{(pse_stress}\OperatorTok{$}\NormalTok{Position)))}
\end{Highlighting}
\end{Shaded}

\includegraphics{phonetics_paper_files/figure-latex/duration-final-1.pdf}

The fairly simple table below shows the mean duration for penult and
final syllables. It is clear that the ultima is longer.

\begin{Shaded}
\begin{Highlighting}[]
\NormalTok{pse_stress }\OperatorTok\StringTok{ }
\StringTok{  }\KeywordTok{group_by}\NormalTok{(Position) }\OperatorTok
\StringTok{  }\KeywordTok{summarize}\NormalTok{(}\DataTypeTok{Duration =} \KeywordTok{round}\NormalTok{(}\KeywordTok{mean}\NormalTok{(Duration), }\DecValTok{3}\NormalTok{)) }\OperatorTok
\StringTok{  }\KeywordTok{kable}\NormalTok{()}
\end{Highlighting}
\end{Shaded}

\begin{tabular}{l|r}
\hline
Position & Duration\\
\hline
penult & 0.070\\
\hline
ultima & 0.152\\
\hline
\end{tabular}

This next table is a bit more complicated and presents the mean duration
of each syllable in sentence final and sentence medial position.

This more complex table below demonstrates the mean duration for each of
the six participants in both sentence final and sentence medial
positions.

\subsection{Intensity}\label{intensity}

The interaction between sentence position and intensity is very similar
across speakers. Speakers show the same pattern: the ultima is more
intense when it is in non-final sentence position.

\begin{Shaded}
\begin{Highlighting}[]
\CommentTok{# A simple boxplot of intensity}
\NormalTok{int <-}\StringTok{ }\KeywordTok{ggplot}\NormalTok{(pse_stress, }\KeywordTok{aes}\NormalTok{(Position, Intensity, }\DataTypeTok{fill=}\NormalTok{Final))}
\NormalTok{int }\OperatorTok{+}\StringTok{ }\KeywordTok{geom_boxplot}\NormalTok{() }\OperatorTok{+}\StringTok{ }\KeywordTok{facet_wrap}\NormalTok{(}\OperatorTok{~}\NormalTok{Participant)}
\end{Highlighting}
\end{Shaded}

\includegraphics{phonetics_paper_files/figure-latex/intensity-1.pdf}

\subsection{Pitch}\label{pitch}

\section{Conclusion}\label{conclusion}

There are some interesting data here, but we cannot conclude much at
this point.

\section*{References}\label{references}
\addcontentsline{toc}{section}{References}

\hypertarget{refs}{}
\hypertarget{ref-adisasmito1996phonetic}{}
Adisasmito-Smith, Niken, and Abigal C. Cohn. 1996. ``Phonetic Correlates
of Primary and Secondary Stress in Indonesian: A Preliminary Study.''
\emph{Working Papers of the Cornell Phonetics Laboratory} 11: 1--15.

\hypertarget{ref-goedemans2007stress}{}
Goedemans, Robertus Wilhelmus Nicolaas, and Ellen van Zanten. 2007.
``Stress and Accent in Indonesian.'' In \emph{Prosody in Indonesian
Languages}, edited by Vincent J van Heuven and Ellen van Zanten, 35--62.
Utrecht: LOT.

\hypertarget{ref-mcdonnell2008conservative}{}
McDonnell, Bradley. 2008. ``A Conservative Vowel Phoneme Inventory of
Sumatra: The Case of Besemah.'' \emph{Oceanic Linguistics} 47 (2):
409--32.

\hypertarget{ref-mcdonnell2016acoustic}{}
---------. 2016. ``Acoustic Correlates of Stress in Besemah.'' In
\emph{NUSA: Linguistic Studies of Languages in and Around Indonesia,
Studies in Language Typology and Change}, edited by Timothy McKinnon and
Yanti, 60:1--28.

\hypertarget{ref-roberts2012}{}
Roberts, Craige. 2012. ``Information Structure in Discourse: Towards an
Integrated Formal Theory of Pragmatics.'' \emph{Semantics and
Pragmatics} 5 (6): 1--69.
doi:\href{https://doi.org/10.3765/sp.5.6}{10.3765/sp.5.6}.

\hypertarget{ref-schwarzschild1999}{}
Schwarzschild, Roger. 1999. ``GIVENness, \textsc{AvoidF} and Other
Constraints on the Placement of Accent.'' \emph{Natural Language
Semantics} 7: 141--77.

\hypertarget{ref-vanzanten2004word}{}
van Zanten, Ellen, and Vincent J. van Heuven. 2004. ``Word Stress in
Indonesian: Fixed or Free.'' \emph{NUSA, Linguistic Studies of
Indonesian and Other Languages in Indonesia} 53: 1--20.

\hypertarget{ref-vanzanten2003status}{}
van Zanten, Ellen, Robertus Wilhelmus Nicolaas Goedemans, and Jos J.A.
Pacilly. 2003. ``The Status of Word Stress in Indonesian.'' In \emph{The
Phonological Spectrum: Suprasegmental Structure}, edited by Jeroen van
de Weijer, Vincent J. van Heuven, and Harry van der Hulst, 151--75.
Amsterdam: John Benjamins.


\end{document}
